\chapter{对某些对象的排版要求}
\lstset{language={[LaTeX]TeX}}
本章介绍的是关于几个常用对象的排版要求。这些要求的效果属于常见的排版效果,于是公司模板中不需要自定义的宏包,只在本章做些要求或建议。为帮助理解这些要求和建议,本章会对一些基本概念做简单的解释,而这些解释不足以向新手介绍清楚对象的用法。

\section{用户自己增加宏包}
如果用户需要增加新的宏包,可以在 preamble 区(\filename{\textbackslash documentclass} 语句之后,\filename{\textbackslash begin\{document\}} 语句之前)加入语句\filename{\textbackslash RequirePackage\{Package\}}。不建议用\filename{\textbackslash usepackage}。

\section{交叉引用和参考文献}
\LaTeX 排版的一个优点是交叉引用的逻辑清晰而严谨,以下做一个简单的介绍。

\subsection{交叉引用}
所有有\filename{Label} 的表格、图片、代码、公式和章节都可以被引用,语法为\filename{\textbackslash ref\{Label\}}。举例:如果想实现“如表~\ref{triline} 所示”字样,\LaTeX 语法为
\footnotesize\begin{verbatim}
如表~\ref{triline} 所示
\end{verbatim}
\normalsize
中间的波浪线的功能是实现“表”字与编号 \ref{triline} 之间的留空。这个留空不用普通的空格来实现,是为了避免在这里换行。

如果想提到某一\filename{Label}对象所在的页码,语法为\filename{\textbackslash pageref\{Label\}}。如果想给某些章节加\filename{Label},方法是在章节名如 \filename{\textbackslash chapter\{Chapter\}}后加一句\filename{\textbackslash label\{Label\}}。\filename{Label} 可能会非常多,作者自己会记不请。对此可参考的建议有:
\begin{itemize}
\item 不需要引用的就不用给\filename{Label},等到需要的时候再给也不迟。
\item 分类取名,比如表格都用\filename{tab:}开头,图片都用\filename{fig:}开头。
\item 用比较清晰的英文简称作为\filename{Label}。
\end{itemize}

\subsection{参考文献及引用}
建议参考文献用\command{bibtex} 来编译。参考文献集中在文件 \filename{main.bib}中,文献格式示例:
\begin{lstlisting}[caption={参考文献格式示例}]
@article{katz1979history,
  title={The History of Stokes's Theorem},
  author={Victor J. Katz},
  journal={Mathematics Magazine},
  volume={52},
  pages={146-156},
  year={1979},
}
\end{lstlisting}
Label 在其中第一行。

这篇文献的引用方法为:
\begin{lstlisting}
\cite\{katz1979history\}
\end{lstlisting}

\section{数学环境}
\LaTeX 是数学系的至爱,因为其数学公式排版漂亮。\LaTeX 有三种数学环境:
\begin{enumerate}
\item math:行内数学环境,像在正文行内出现 $E=mc^2$ 这样。
\item displaymath:独立占行,不参与公式编号,无\filename{Label}。
\item equation:独立占行,参与公式编号,可以设\filename{Label}来引用。
\end{enumerate}

我们要求所有的数学内容都用数学环境,包括物理量。比如提到深度变量的时候用 $d$ 而不是 d,前者变用了 math 环境。排出漂亮的公式需要不少数学语法,可以准备网页版的参考手册,也可以考虑 lyx 或其他软件来生成。

\begin{paraitem}
\item math 环境的语法为两端用\textbackslash(和\textbackslash)括起来(或者用\$括起来)。例如质能方程的语法为\textbackslash(E=mc\^{}2\textbackslash),也可以是\$E=mc\^{}2\$。

\item display 环境的语法是两端用\textbackslash[和\textbackslash]括起来。(也可以用 \$\$ 括,但据介绍可能会与其他宏包冲突,不推荐。)举例:\[\int_0^\infty \mathrm{e}^{-x}\,\mathrm{d}x\]
其语法为:
\begin{lstlisting}
\[ \int_0^\infty \mathrm{e}^{-x}\,\mathrm{d}x \]
\end{lstlisting}

\item equation 环境来表达同一个积分式,效果为:
\begin{equation}\label{eq:integral}
\int_0^\infty \mathrm{e}^{-x}\,\mathrm{d}x
\end{equation}
式~\ref{eq:integral} 的语法为:
\begin{lstlisting}
\begin{equation}\label{eq:integral}
\int_0^\infty \mathrm{e}^{-x}\,\mathrm{d}x
\end{equation}
\end{lstlisting}
\end{paraitem}

需要注意的是,在数学模式中,一般的字母会被解释为变量,于是变成了斜体,当你不需要斜体的时候,就要作特别声明,如\filename{\textbackslash mathrm}。也有宏包做其他的定义方式,$\diff x$ 中的 $\diff$ 可以用\filename{\textbackslash diff};$r=\unit{2.5cm}$ 中的 $\unit{2.5cm}$ 可以用\filename{\textbackslash unit\{2.5cm\}}。


\section{算法}
有些宏包可以帮助\LaTeX 排版算法(伪代码)。这里举一个例子:
\begin{algorithmic}
\If {$i\geq maxval$}
    \State $i\gets 0$
\Else
    \If {$i+k\leq maxval$}
        \State $i\gets i+k$
    \EndIf
\EndIf
\end{algorithmic}

这个例子的代码为(用了\filename{algpseudocode} 宏包):
\begin{lstlisting}
\begin{algorithmic}
\If {$i\geq maxval$}
    \State $i\gets 0$
\Else
    \If {$i+k\leq maxval$}
        \State $i\gets i+k$
    \EndIf
\EndIf
\end{algorithmic}
\end{lstlisting}
