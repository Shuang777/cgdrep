\chapter{模板的格式要求}
我们暂定了表格、图片、代码和其他一些对象的格式要求,并写了一些宏放在模板里。目前这些格式要求并不是最终版,欢迎提出意见。

一般,每一个表格(图片、代码)都当有一对\filename{Caption} 和\filename{Label}。\filename{Caption} 是这个表格(图片、代码)的名字,其字号比正文小半号。\filename{Label} 是引用它时用的标识,\filename{Label} 本身的内容并不出现在排版后的\filename{pdf}页面上。以下分别讲如何实现表格、图片和代码格式的实现方法。

\section{表格}
用户没有特别需要的话,我们希望表格都采用三线表,首行为其\filename{header},用黑体。字号比正文小半号,位置居中。如表~\ref{triline}:
\begin{ctable}{triline}{三线表}{llll}
 版本 & 日期       & 负责人       & 备注 \\ \hline 
 1.0  & 2013.12.24 & 沈忱、纪冬梅 & 初稿 \\
 1.1  & 2013.12.25 & 赵军         & 二稿 \\
\end{ctable}

实现这样的表及其浮动位置,可以用公司自己制作的宏 \filename{ctable},其语法为:
\begin{lstlisting}[language={[LaTeX]TeX},caption={ctable 语法}]
\begin{ctable}{Label}{Caption}{Alignments}
 第一行(header)  \\ \hline 
 第二行 \\
.....
\end{ctable}\end{lstlisting}

其中\filename{Alignments}是列对齐方式,四列左对齐为\filename{\{llll\}}。则表~\ref{triline} 的完整语法为:
\begin{lstlisting}[language={[LaTeX]TeX},caption={ctable 示例}]
\begin{ctable}{triline}{三线表}{llll}
 版本 & 日期       & 负责人       & 备注 \\ \hline 
 1.0  & 2013.12.24 & 沈忱、纪冬梅 & 初稿 \\
 1.1  & 2013.12.25 & 赵军         & 二稿 \\
\end{ctable}
\end{lstlisting}

如果你不想局限于 \filename{ctable} 的功能,也可以用一般的 \LaTeX 表格语法(模板包含了 \filename{tabu} 宏包),表~\ref{triline} 的一般\LaTeX 语法为:
\begin{lstlisting}[language={[LaTeX]TeX},caption={三线表的一般\LaTeX 语法示例}]
\begin{table}[htbp]\caption{\label{triline} 三线表}
\centering\small\begin{tabu}{llll}\thickhline\rowfont{\bfseries}
 版本  & 日期       & 负责人      & 备注 \\ \hline
 1.0  & 2013.12.24 & 沈忱、纪冬梅 & 初稿 \\
 1.1  & 2013.12.25 & 赵军         & 二稿 \\
\thickhline\end{tabu}\end{table}
\end{lstlisting}

\section{图片}
插入图片的语法简单,这里没有设置自己的宏,而是用的是普通语法:
\begin{lstlisting}[language={[LaTeX]TeX},caption={插入图片语法示例}]
\begin{figure}[htbp]\centering
\includegraphics[width=0.5\textwidth]{file.jpg} 
\caption{\label{Label}Caption}
\end{figure}\end{lstlisting}
对于图片的大小控制,当用户没有特别需求的时候,建议采用\filename{textwidth}的倍数方式。

\section{代码}
这里先给出一段文档中插入 Pascal 代码的样子:
\begin{lstlisting}[language=Pascal,caption={Pascal 代码示例},label=PascalExample]
for i:=maxint to 0 do
begin
{ do nothing }
end;
Write(’Case insensitive ’);
Write(’Pascal keywords.’);
\end{lstlisting}

插入代码的 \LaTeX 语法为:
\begin{lstlisting}[language={[LaTeX]TeX},caption={插入代码的语法}]
\begin{lstlisting}[language=Language,caption={Caption},label=Label]
.......
 (代码内容)
.......
\end{lstlisting }\end{lstlisting}

其中\filename{Languge}为这一段代码的语言。支持的语言包括 ABAP、ACSL、Ada、Algol、Ant、Assembler、Awk、bash、Basic、C、C++、Caml、Clean、Cobol、Comal、csh、Delphi、Eiffel、Elan、erlang、Euphoria、Fortran、GCL、Gnuplot、Haskell、HTML、IDL、inform、Java、JVMIS、ksh、Lisp、Logo、make、Mathematica、Matlab、Mercury、MetaPost、Miranda、Mizar、ML、Modelica、Modula-2、MuPAD、NASTRAN、Oberon-2、OCL、Octave、Oz、Pascal、Perl、PHP、PL/I、Plasm、POV、Prolog、Promela、Python、R、Reduce、Rexx、RSL、Ruby、S、SAS、Scilab、sh、SHELXL、Simula、SQL、tcl、TeX、VBScript、Verilog、VHDL、VRML、XML、XSLT。更多信息请搜索关键词 lstlisting。代码~\ref{PascalExample} 的引用方法为:
\begin{lstlisting}[language={[LaTeX]TeX},caption={插入代码示例}]
\begin{lstlisting}[language=Pascal,caption={Pascal 示例代码},label=PascalExample]
for i:=maxint to 0 do
begin
{ do nothing }
end;
Write(’Case insensitive ’);
Write(’Pascal keywords.’);
\end{lstlisting }
\end{lstlisting}

目前暂定代码的文字大小为 \filename{\textbackslash footnotesize}。如果用户需要调整,可以在引用前修改字体,比如大半号为\filename{\textbackslash small},其语法为:
\begin{lstlisting}[language={[LaTeX]TeX}]
\lstset{basicstyle=\small\ttfamily}
\end{lstlisting}
这个命令将作用于后面所有的代码。如果这并不是用户想要的,则需要修改回来:
\begin{lstlisting}[language={[LaTeX]TeX}]
\lstset{basicstyle=\footnotesize\ttfamily}
\end{lstlisting}
同理,用户也可以一次设定语言:
\begin{lstlisting}[language={[LaTeX]TeX}]
\lstset{languge=Language}
\end{lstlisting}
则后面每次插入代码时候就不用加入\filename{language=Languge}这一句了。更详细的用法请搜索关键词 lstlisting。

如果代码比较长,在某个原始代码文件里,可以用如下语法,将整个文件的代码引入:
\begin{lstlisting}[language={[LaTeX]TeX},caption={引入代码文件示例}]
\lstinputlisting[language=Language,label=Label,caption={Caption}]{Filename}
\end{lstlisting}

下面给出 C、Python 和 Bash 代码示例,欢迎提出排版意见。

\begin{lstlisting}[language=C,caption={C 代码示例},label=cLabel]
#include <stdio.h>
#define N 10
/* Block
 * comment */
 
int main()
{
    int i;
 
    // Line comment.
    puts("Hello world!");
 
    for (i = 0; i < N; i++)
    {
        puts("LaTeX is also great for programmers!");
    }
 
    return 0;
}
\end{lstlisting}

\begin{lstlisting}[language=python,caption={Python 代码示例},label=Python]
class BankAccount(object):
    def __init__(self, initial_balance=0):
        self.balance = initial_balance
    def deposit(self, amount):
        self.balance += amount
    def withdraw(self, amount):
        self.balance -= amount
    def overdrawn(self):
        return self.balance < 0
my_account = BankAccount(15)
my_account.withdraw(5)
print my_account.balance
import unittest
def median(pool):
    copy = sorted(pool)
    size = len(copy)
    if size % 2 == 1:
        return copy[(size - 1) / 2]
    else:
        return (copy[size/2 - 1] + copy[size/2]) / 2
class TestMedian(unittest.TestCase):
    def testMedian(self):
        self.failUnlessEqual(median([2, 9, 9, 7, 9, 2, 4, 5, 8]), 7)
if __name__ == '__main__':
    unittest.main()
\end{lstlisting}

\lstinputlisting[language=bash,label=bash.sh, caption={bash 代码示例}]{bash1.sh}

\section{其他格式要求}
\subsection{几个字体宏}
有几个特别对象,要在文档里用特别的字体来,我们定义了几个字体的宏,如表~\ref{special} 所列。文档中用到这样的对象时,请采用表中的字体宏。

\begin{ctable}{special}{几类特别对象}{lll} 
对象 & 语法示例 & 效果示例 \\ \hline
文件名 & \textbackslash filename\{directory/file.name\} & \filename{directory/file.name} \\
命令名 & \textbackslash command\{xelatex main\} & \command{xelatex main} \\
参数名 & \textbackslash parameter\{Energy\} & \parameter{Energy} \\
用户输入 & \textbackslash userinput\{runApp\} & \userinput{runApp} \\
图形界面菜单 & \textbackslash guimenu\{File > Open\} & \guimenu{File>Open} \\
\end{ctable}

\subsection{Item 列表}
对于三个 item 列表(itemize、enumerate和description),我们都改了间距。如果用户想在每个 item 里放一段落,于是想保留段落间距的话,可以用新的环境变量 paraitem。语法为:
\begin{lstlisting}[language={[LaTeX]TeX},caption={保留段落间距的环境变量}]
\begin{paraitem}
\item 第一段。
\item 第二段。
.....
\end{paraitem}
\end{lstlisting}

