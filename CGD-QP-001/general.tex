\chapter{要求}

为规范公司文档的管理,特制定本办法。目前为征求意见阶段,欢迎意见和建议。
\section{适用范围}
本管理办法适用于所有权归公司的、密级为“内部”及“公开”的正式文档。包括但不限于:
\begin{itemize}
\item 公司对外的正式报告,包括技术方案、技术总结以及操作手册等。
\item 公司对外的书面交流文档。
\item 公司对内的质量流程、开发文档以及有存档价值的过程文档等。
\end{itemize}
不属于此办法的管理范围:
\begin{itemize}
\item 密级为“秘密”及以上的文档。
\item 无存档价值的交流文档。
\end{itemize}

\section{排版要求}
公司要求以后的正式文档尽可能用 \LaTeX 排版,并使用公司统一的模板。具体要求和实现方法见本文档后面的部分。

\section{文件存档}

\subsection{应有文件}
每一份文档对应的不是一个文件,而是一个{\bf 目录}。这个目录中应有两个文件:
\begin{enumerate}
\item 最终的 \filename{pdf} 文件。
\item 一个 \filename{zip} 或 \filename{gzip} 包,其中包含生成这个 \filename{pdf} 文件所需要的所有 \filename{tex} 文件、图片文件及其他文件如插入代码、原始数据等。
\end{enumerate}
注意事项:
\begin{enumerate}
\item 如果这个文档是用 lyx 软件编辑的,则应在 \filename{zip}/\filename{gzip} 压缩包中包括 \filename{lyx} 文件。
\item 如果文档中的插图是用 LibreOffice 等软件制作的,应在 \filename{zip}/\filename{gzip} 压缩包中包括 \filename{odt}/\filename{odp} 等文件。
\item 如果生成 \filename{pdf} 文档需要特殊的操作步骤,或有其他需要注意的事项,应当在 \filename{README} 文件中加以说明。
\end{enumerate}

\subsection{文件位置}
存在服务器上,全公司可读,总路径为:\filename{/home/public/document}。以本文档为例,其完整的目录为:\filename{/home/public/document/procedure/CGD-QP-001/V01/}。其三层路径名:
\begin{itemize}
\item \filename{procedure} 表示文档类型为质量流程,
\item \filename{CGD-QP-001} 为本文档的编号,
\item \filename{V01} 表示版本。
\end{itemize}
这个路径里面有两个文件,一个 \filename{pdf} 文件,一个原始文件的 \filename{gzip} 包。

\subsection{文档编号规则}
文档编号为 CGD-XX-xxx,其中 CGD 表示 Cogenda,XX 为字母,表示文档类型。xxx 为数字,为本文档的顺序号。编号 XX 对应的类型及其存放路径为:

\begin{tabular}{lll} \thickhline
\bf XX & \bf 文档类型 & \bf 存放路径 \\ \hline
TP & 技术方案 & \filename{proposal} \\
TR & 技术总结 & \filename{report} \\
MN & 操作手册 & \filename{manual} \\
QP & 质量流程 & \filename{procedure} \\
MS & 其他文档 & \filename{miscellaneous} \\
\thickhline \end{tabular}

\subsection{目录名和文件名}
每个文档对应的目录名是文档的完整编号,不可以用其他内容。
文档内的文件名可以自由命名,但只能由英文字母、数字、下划线(\_)或英文句点(.)组成,不可以出现其他字符。

\subsection{入档}
文件入档,如果还没有文档编号,请先跟公司文档管理员申请编号,填入文档后在 helium 下测试是否顺利生成 \filename{pdf} 文档,然后将文件拷给公司文档管理人员。

% 文件查询,即想找到文档对应的编号,可以打开文件 \filename{/home/public/document/index.html}来查询。
